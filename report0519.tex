\documentclass{amsart}
\synctex=1

%=================================================================
% 
\newcount\DraftStatus  % 0 suppresses notes to selves in text
\DraftStatus=1   % TODO: set to 0 for final version
%=================================================================

%=================================================================
\usepackage{comment}
%=================================================================
%
\includecomment{JournalOnly}  
\includecomment{ConferenceOnly}  
\includecomment{TulipStyle}
%
%=================================================================
\input{preamble}


%=================================================================
%
\begin{document}
%
%=================================================================
%
\title[Learning Process]{Learning progress report}%

\author{PengCheng Jiang}
\address[A.~1]{School of Computer Science,\\ 
JiLin University, jilin 130000, China}%
\email[A.~1]{pcjiang@tulip.academy}


%\thanks{Thanks to \ldots}%
\subjclass{Artificial Intelligence}%
\date{\gitAuthorDate}%


\maketitle
\tableofcontents

\newpage
%=================================================================

%=================================================================
\section{Introduction to machine learning}
    \begin{itemize}
        \item The process of machine learning
        \item Why use machine learning
        \item Application scenarios of machine learning
        \item Three key problems of machine learning       
    \end{itemize}    
\section{linear regression}
    \begin{itemize}
        \item The concept of linear regression
        \item Loss function of logistic regression
            \begin{itemize}
                \item mean square error
            \end{itemize}
        \item The solution of logistic regression
            \begin{itemize}
                \item Gradient descent method
                \item least square method(Matrix of full rank,maximum likelihood)
            \end{itemize}
        \item Methods to solve the underfitting and overfitting
    \end{itemize}   
\section{logistic regression}
    \begin{itemize}
        \item Logistic distribution and density function
        \item The concept of logistic regression
        \item Loss function of logistic regression
            \begin{itemize}
                \item maximum likelihood
            \end{itemize}
        \item The solution of logistic regression
            \begin{itemize}
                \item Gradient descent method
                \item Newton method
            \end{itemize}   
        \item The method and significance of regularization
            \begin{itemize}
                \item L1 regularization: LASSO regularization
                \item L2 regularization: Ridge regularization
                \item the difference between L1 and L2
            \end{itemize}
        \item Parallelization of gradient computation of objective function
        \item Comparison with other models
            \begin{itemize}
                \item linear regression
                \item maximum entropy model
                \item SVM
                \item Naive Bayes
            \end{itemize}
    \end{itemize}
\section{K-nearest neighbor model}
\begin{itemize}
    \item concept
        \begin{itemize}
            \item Distance measurement
            \item Selection of K value
            \item Classification decision rules
        \end{itemize}
    \item kd tree
        \begin{itemize}
            \item The construction of KD tree
            \item The search of KD tree
        \end{itemize}
\end{itemize}
\end{document}

